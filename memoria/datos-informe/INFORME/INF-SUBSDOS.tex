\subsection {Denegaci�n de Servicios (DoS).} 

Las redes WLAN son propensas a los ataques de denegaci�n de servicio (DoS) usando varias t�cnicas, incluyendo pero no limitadas a: \\

 \begin{itemize}
   	\item Ataque de de-autenticaci�n 
	\item Ataque de desasociaci�n.
	\item Ataque CTS-RTS.
	\item Ataque de interferencia del espectro de la se�al. 
 \end{itemize}

El objetivo de una Denegaci�n de Servicio a una red WiFi es dejar a los usuarios leg�timos de una red WiFi sin poder acceder a Internet, esto se logra inundando con paquetes de de-autenticaci�n al punto de acceso AP y/o al cliente.\\

Continuamos trabajando con la tarjeta en modo monitor, hemos asignando el canal del punto de acceso AP que estamos analizando.\\

\textbf{airmon-ng start wlan0 11}\\

Luego realizamos el env�o de difusi�n de de-autenticaci�n de paquetes (broadcast de-authentication packet) hacia el punto de acceso AP intentando desconectar a todos los clientes.\\

\textbf{aireplay-ng -0 0 -a 80:B6:86:D5:F9:4B wlan0mon}\\

\textbf{aireplay-ng -0 0 -a 80:B6:86:D5:F9:4B -c AC:38:70:25:ED:62 wlan0mon}\\

Hemos conseguido enviar con �xito frames de de-autenticaci�n al punto de acceso y al cliente. Esto se ha traducido en conseguir que se desconecte y se pierda la comunicaci�n entre ellos.
Tambi�n hemos enviado paquetes de difusi�n de de-autenticaci�n, que asegurar� que ning�n cliente en las cercan�as se pueda conectar correctamente al punto de acceso.\\

Es importante tener en cuenta que tan pronto como el cliente se desconecta, intentar� volver a conectarse de nuevo al punto de acceso y por lo tanto el ataque de de-autenticaci�n tiene que llevarse a cabo de manera sostenida para tener un efecto de ataque de denegaci�n de servicio completo.\\