\subsection{Ataque a red Wifi con cifrado WEP}

Para obtener la clave WEP de un punto de acceso, necesitamos muchos vectores de inicializaci�n (IVs). El tr�fico de red habitual no genera de forma r�pida suficientes IVs. Para ello hemos utilizado la t�cnica de inyecci�n para aumentar la velocidad del proceso de captura. La inyecci�n implica que se env�en al punto de acceso (AP) paquetes de forma continua y r�pida permitiendo capturar un gran n�mero de IV's en un periodo corto de tiempo. Una vez que se han capturado un gran n�mero de IVs, podemos utilizarlos para averiguar la clave WEP.

Para tratar de obtener la clave se ha realizado el ataque est�ndar utilizando los siguientes pasos:\\


Hemos colocado nuestra tarjeta en modo monitor y fijado al canal del AP.\\

\textbf{airmon-ng start wlan0 6}\\

A continuaci�n se ha utilizado el comando \textit{`airodump-ng'} en el canal del AP con filtro de bssid para capturar los IVs.\\

\textbf{airodump-ng -c 6 --bssid 80:B6:86:D5:F9:4B -w output wlan0mon}\\


Con el comando \textit{`aireplay-ng'} se procede a desautenticar a un cliente asociado con el fin de que vuelva a autenticarse y genere un paquete ARP v�lido.\\

\textbf{aireplay-ng -0 5 -a 80:B6:86:D5:F9:4B -c AC:38:70:25:ED:62 wlan0mon}\\

Para generar mucho paquetes y conseguir gran cantidad de IVs se lanza una reinyecci�n de paquetes utilizando el siguiente comando:\\

\textbf{aireplay-ng -3 -b 80:B6:86:D5:F9:4B -h AC:38:70:25:ED:62 wlan0mon}\\

Finalmente y si hemos conseguido una gran cantidad de IVs intentamos probar conseguir la clave WEP utilizando el comando \textit{`aricrack-ng'}.\\

\textbf{aircrack-ng -a 1 -s output-01.cap}\\

En este an�lisis de red se ha conseguido \textbf{39885} IVs y \textbf{SI} se ha podido obtener la clave.

\textbf{}\\

El detalle de la salida de Aircrack-ng es el siguiente:

\begingroup
  \fontsize{7pt}{7pt}=\selectfont
  \verbatiminput{../AIRCRACK-WEP/aircrack-80:B6:86:D5:F9:4B.dat}
\endgroup

