\subsection{Rompiendo Contraseña.}

Buscamos descifrar la contraseña de acceso al dispositivo emisor de la señal inalámbrica. Con ello podremos realizar conexión fraudulenta al mismo, para tener acceso a la red inalámbrica y servicios adicionales presentes como por ejemplo, Bases de Datos e Intranet Corporativa, u obtener información sensible. La prueba comprende cuatro etapas: Exploración, Interceptación, Inyección y Descifrado. Para llevarla a cabo, se requiere una tarjeta de red inalámbrica con capacidad de inyección, así como la suite de aplicaciones de Aircrack, incluida en la distribución Kali Linux.

Primeramente iniciamos nuestra interfaz wifi sobre el canal AP que analizamos.\\

\textbf{\VAR{x1}}\\

A continuación intentamos capturar los 4 paquetes del handshake en el momento que un cliente se autentifica con el AP que estamos analizando.\\

\textbf{\VAR{x2}}\\

Luego utilizamos aireplay-ng para deauntetificar al cliente conectado. Intentamos enviar un mensaje al cliente para desasociarlo de la AP que estamos analizando.\\

\textbf{\VAR{x3}}\\

Finalmente intentamos conseguir la clave WPA/WPA2 pre-compartida utilizando aircrack-ng y con la ayuda de un diccionario de posibles palabras. Básicamente aircrack-ng compruena cada una de las palabras si coincide con la clave.\\

\textbf{\VAR{x4}}\\

Este análisis se ha realizado en un tiempo de \textbf{\VAR{x5}} segundos, y \textbf{\VAR{x6}} se ha podido encontrar la clave. El detalle es el siguiente.\\

\begingroup
        \fontsize{7pt}{7pt}=\selectfont
        \verbatiminput{../AIRCRACK-WPA/\VAR{x7}}
\endgroup

