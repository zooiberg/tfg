\subsection{Ataque Reaver-Pixiewps}

Mediante la utilizaci�n de \textit{reaver} llevamos a cabo un ataque de fuerza bruta contra el n�mero pin de la configuraci�n protegida del punto de acceso WiFi. Una vez que el pin WPS es encontrado, la WPA PSK puede ser recuperada y alternativamente la configuraci�n inal�mbrica del AP puede ser reconfigurada.\\

Adem�s de \textit{reaver}, si es posible se utilizar� la herramientas \textit{pixiewps} para realizar an�lisis de fuerza bruta del PIN WPS en el dispositivo AP que analizamos por si su entrop�a es nula o d�bil (ataque pixie dust).\\

Iniciamos el ataque, el cual no es por pines donde el sistema testea todas las combinaciones posibles de un grupo de 8 d�gitos en caso de haber otra combinaci�n se demora de 1 a 2 horas que necesita para acceder a la clave.\\

\textbf{reaver -F -G -i wlan0mon -b 80:B6:86:D5:F9:4B -c 11 -a -n -vv -D}\\

El resultado obtenido es: \textbf{NO SE HAN PODIDO OBTENER LOS DATOS NECESARIOS DE 80:B6:86:D5:F9:4B}. A continuaci�n el detalle de los datos capturados.\\

\begingroup
        \fontsize{7pt}{7pt}=\selectfont
        \verbatiminput{../LOGS/TXT_ATAQUE}
\endgroup

\input{FIN-Template}
