\subsection{Rompiendo Contrase�a.}

Buscamos descifrar la contrase�a de acceso al dispositivo emisor de la se�al inal�mbrica. Con ello podremos realizar conexi�n fraudulenta al mismo, para tener acceso a la red inal�mbrica y servicios adicionales presentes como por ejemplo, Bases de Datos e Intranet Corporativa, u obtener informaci�n sensible. La prueba comprende cuatro etapas: Exploraci�n, Interceptaci�n, Inyecci�n y Descifrado. Para llevarla a cabo, se requiere una tarjeta de red inal�mbrica con capacidad de inyecci�n, as� como la suite de aplicaciones de `Aircrack', incluida en la distribuci�n Kali Linux.

Primeramente iniciamos nuestra interfaz WiFi sobre el canal AP que analizamos.\\

\textbf{airmon-ng start wlan0 11}\\

A continuaci�n intentamos capturar los 4 paquetes del handshake en el momento que un cliente se autentifica con el AP que estamos analizando.\\

\textbf{airodump-ng -c 11 --bssid 80:B6:86:D5:F9:4B -w output wlan0mon}\\

Luego utilizamos \textit{`aireplay-ng'} para deauntetificar al cliente conectado. Intentamos enviar un mensaje al cliente para desasociarlo de la AP que estamos analizando.\\

\textbf{aireplay-ng -0 1 -a 80:B6:86:D5:F9:4B -c AC:38:70:25:ED:62 wlan0mon}\\

Finalmente intentamos conseguir la clave WPA/WPA2 pre-compartida utilizando \textit{`aircrack-ng'} y con la ayuda de un diccionario de posibles palabras. B�sicamente \textit{`aircrack-ng'} comprueba cada una de las palabras si coincide con la clave.\\

\textbf{aircrack-ng -a 2 -s output-01.cap -w wordlist.lst}\\

Este an�lisis se ha realizado en un tiempo de \textbf{90} segundos, y \textbf{SI} se ha podido encontrar la clave. El detalle es el siguiente.\\

\begingroup
        \fontsize{7pt}{7pt}=\selectfont
        \verbatiminput{../AIRCRACK/aircrack-80:B6:86:D5:F9:4B.dat}
\endgroup
