\subsection {Denegacion de Servicios (DoS).} 
Las redes WLAN son propensas a los ataques de denegaci�n de servicio (DoS) usando varias t�cnicas, incluyendo pero no limitadas a: 
 \begin{itemize}
   	\item Ataque de de-autenticaci�n 
	\item Ataque de desasociaci�n.
	\item Ataque CTS-RTS.
	\item Ataque de interferencia del espectro de la se�al. 
 \end{itemize}

El objetivo de una Denegaci�n de Servicio a una Red Wi-fi es dejar a los usuarios leg�timos de una red Wi-fi sin poder acceder a Internet, esto se logra inundando con paquetes de deautenticaci�n al punto de acceso y/o al cliente.
Cuando se quiere realizar un ataque DoS a una Red Wi-fi, se puede atacar de 2 formas, una es atacando al punto de acceso y dejar sin acceso a ning�n cliente o atacando solamente a un cliente espec�fico.\\

Trabajando con la tarjeta en modo monitor y asignando el canal de la red en la cual vamos a anaizar.\\

\textbf{airmon-ng start wlan0 11}\\

A continuaci�n realizamos el env�o de difusi�n de de-autenticaci�n de paquetes (broadcast de-authentication packet) hacia el punto de acceso de la red inal�mbrica, intentamos desconectar a todos los clientes.\\

\textbf{aireplay-ng -0 0 -a 80:B6:86:D5:F9:4B wlan0mon}\\

\textbf{aireplay-ng -0 0 -a 80:B6:86:D5:F9:4B -c 00:13:EF:71:02:DD wlan0mon}\\

Si hemos conseguido enviar con �xito frames de de-autenticaci�n al punto de acceso y el cliente. Esto se ha traducido en conseguir que se desconecte y una p�rdida completa de comunicaci�n entre ellos.
Tambi�n hemos enviado paquetes de difusi�n de de-autenticaci�n, que asegurar� que ning�n cliente en las cercan�as se pueda conectar correctamente al punto de acceso.
Es importante tener en cuenta que tan pronto como el cliente se desconecta, intentar� volver a conectarse de nuevo al punto de acceso y por lo tanto el ataque de de-autenticaci�n tiene que llevarse a cabo de manera sostenida para tener un efecto de ataque de denegaci�n de servicio completo. 
