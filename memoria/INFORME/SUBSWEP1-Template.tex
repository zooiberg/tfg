\paragraph{Ataque a red Wifi con cifrado WEP}

Para obtener la clave WEP de un punto de acceso, necesitamos muchos vectores de inicializaci�n (IVs). El tr�fico de red habitual no genera de forma rápida suficientes IVs. Para ello hemos utilizado la t�cnica de inyecci�n para aumentar la velocidad del proceso de captura. La inyecci�n implica que se env�en al punto de acceso (AP) paquetes de forma continua y rápida permitiendo capturar un gran n�mero de IV's en un periodo corto de tiempo. Una vez que se han capturado un gran n�mero de IVs, podemos utilizarlos para averiguar la clave WEP.

Para tratar de obtener la clave se ha realizado el ataque est�ndar utilizando los siguientes pasos:\\


Hemos colocado nuestra tarjeta en modo monitor y fijado al canal del AP.\\

\textbf{\VAR{x1}}\\

A continuaci�n se ha utilizado el comando airodump-ng en el canal del AP con filtro de bssid para capturar los IVs.\\

\textbf{\VAR{x2}}\\


Con el comando aireplay-ng se procede a desautenticar a un cliente asociado con el fin  de que vuelva a autenticarse y genera un paquete ARP v�lido.\\

\textbf{\VAR{x3}}\\

Con el fin de generar mucho paquetes y conseguir gran cantidad de IVs se lanza una reinyecci�n de paquetes utilizando el siguiente comando:\\

\textbf{\VAR{x4}}\\

Finalmente y si hemos conseguido una gran cantidad de IVs intentamos probar conseguir la clave WEP utilizando el comando aricrack-ng.\\

\textbf{\VAR{x5}}\\

En este an�lisis de red se ha conseguido \textbf{\VAR{x6}} IVs y \textbf{\VAR{x7}} hemos podido conseguir la clave.

\textbf{\VAR{x8}}\\

El detalle de la salida de Aircrack-ng es el siguiente:

\begingroup
        \fontsize{7pt}{7pt}=\selectfont
        \verbatiminput{\VAR{x9}}
\endgroup


