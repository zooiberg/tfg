\subsection{Ataque Reaver (Pixiewps)}
Mediante la utilizci�n de reaver llevamos a cabo un ataque de fuerza bruta contra el n�mero pin de la configuraci�n protegida del punto de acceso wifi. Una vez que el pin WPS es encontrado, la WPA PSK puede ser recuperada y alternativamente la configuraci�n inal�mbrica del AP puede ser reconfigurada.\\

Adem�s de reaver tambi�n utilizaremos la herramientas pixiewps para realizar an�lisis de fuerza bruta del PIN WPS en el dispositivo seleccionado con una entrop�a nula o d�bil.

Iniciamos el ataque, el cual no es por pines donde el sistema testea todas las combinaciones posibles de un grupo de 8 d�gitos en caso de haber otra combinaci�n se demora de 1 a 2 horas) que necesita para acceder a la clave.\\

10\\

El resultado del an�lisis es el siguiente:

\begingroup
        \fontsize{7pt}{7pt}=\selectfont
        \verbatiminput{20}
\endgroup


Luego del an�lisis realizado en 30 segundos, 40 ha logrado encontrar las cadenas necesarias para utilizarlo con pixiewps.

pKe: 1\\
pKr: 2\\
E-Hash1: 3\\
E-Hash2: 4\\
AuthKey: 5\\
e-nonce: 6\\

Con los datos capturados, iniciamos el ataque pixiewps.\\

7\\

Seg�n el resultado del an�lisis, 8 se encontr� el PIN WPS 9.\\

A continuaci�n lanzamos el ataque con reaver "mod" para obtener la clave WPA.

10

Y el resultado es el siguiente:

\begingroup
        \fontsize{7pt}{7pt}=\selectfont
        \verbatiminput{11}
\endgroup
\\


